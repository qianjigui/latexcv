\newlength{\outerbordwidth}
\pagestyle{empty}
\raggedbottom
\raggedright
\usepackage[svgnames]{xcolor}
\usepackage{framed}
\usepackage{tocloft}

% Set link in the document
\usepackage{hyperref}
\hypersetup{colorlinks,%
            citecolor=blue,%
            filecolor=black,%
            linkcolor=red,%
            urlcolor=blue,%
            pdfauthor={\personName(\personEmail)},
            pdftitle={\personName's CV},
            pdfsubject={CV},
            pdfkeywords={CV, IT, WPC},
            pdftex
}

\let \oldhref \href
\renewcommand{\href}[2]{
  \oldhref{#1}{#2:#1}
}

% Support Chinese
\usepackage[slantfont,boldfont,CJKnumber,CJKtextspaces]{xeCJK} % 加载 xeCJK,允许斜体、粗体和 CJK 数字以及 CJK 对空格的设置
%sudo apt-get install ttf-wqy-microhei
%sudo apt-get install ttf-wqy-zenhei
%WenQuanYi Zen Hei,文泉驛正黑,文泉驿正黑:style=Regular
%WenQuanYi Micro Hei,文泉驛微米黑,文泉驿微米黑:style=Regular
%WenQuanYi Zen Hei Mono,文泉驛等寬正黑,文泉驿等宽正黑:style=Regular
%WenQuanYi Zen Hei Sharp,文泉驛點陣正黑,文泉驿点阵正黑:style=Regular

%\setCJKmainfont{WenQuanYi Zen Hei}
\setCJKmainfont[BoldFont={WenQuanYi Micro Hei}, ItalicFont={WenQuanYi Zen Hei Sharp}]{WenQuanYi Zen Hei} %中文缺省字体
\setCJKsansfont{WenQuanYi Zen Hei} %中文无衬线字体
\setCJKmonofont{WenQuanYi Zen Hei Mono} %中文等宽字体

%\setCJKmainfont[BoldFont={Adobe Heiti Std}, ItalicFont={Adobe Kaiti Std}]{Adobe Song Std} %中文缺省字体
%\setCJKsansfont{Adobe Song Std} %中文无衬线字体
%\setCJKmonofont{Adobe Fangsong Std} %中文等宽字体


%-----------------------------------------------------------
%Edit these values as you see fit

\setlength{\outerbordwidth}{3pt}  % Width of border outside of title bars
\definecolor{shadecolor}{gray}{0.75}  % Outer background color of title bars (0 = black, 1 = white)
\definecolor{shadecolorB}{gray}{0.93}  % Inner background color of title bars

%-----------------------------------------------------------
%Margin setup

\setlength{\evensidemargin}{-0.25in}
\setlength{\headheight}{0in}
\setlength{\headsep}{0in}
\setlength{\oddsidemargin}{-0.25in}
\setlength{\paperheight}{11in}
\setlength{\paperwidth}{8.5in}
\setlength{\tabcolsep}{0in}
\setlength{\textheight}{9.5in}
\setlength{\textwidth}{7in}
\setlength{\topmargin}{-0.3in}
\setlength{\topskip}{0in}
\setlength{\voffset}{0.1in}


% Person Environment
\newenvironment{person}[0]{
  %%%%%%%%%%%%%%%%%%%%%%%%%%%%%%
  \resheading{\personHeading}
  %%%%%%%%%%%%%%%%%%%%%%%%%%%%%%
  \begin{center}
}
{
  \end{center}
}
% For Personal items
\newcommand{\personItem}[1]{
  \parbox{6.762in}{#1}
}

% Education Environment
\newenvironment{education}[0]{
%%%%%%%%%%%%%%%%%%%%%%%%%%%%%%
\resheading{\educationHeading}
%%%%%%%%%%%%%%%%%%%%%%%%%%%%%%
\begin{itemize}
}
{
\end{itemize}
}
% Education Item
% 1st: university name
% 2nd: location
% 3td: profession
% 4th: time
% 5th: items
\newcommand{\educationItem}[5]{
\item
  \ressubheading{#1}{#2}{#3}{#4}
  \begin{itemize}
    #5
  \end{itemize}
}

% Work Environment
\newenvironment{works}[1][\workHeading]{
%%%%%%%%%%%%%%%%%%%%%%%%%%%%%%
\resheading{#1}
%%%%%%%%%%%%%%%%%%%%%%%%%%%%%%
\begin{itemize}
}
{
\end{itemize}
}
% Item
% 1: Company name
% 2: Location
% 3: position
% 4: Time
% 5: resitems
\newcommand{\workItem}[5]{
\item
  \ressubheading{#1}{#2}{#3}{#4}
  \begin{itemize}
    #5
  \end{itemize}
}

%Skills Environment
\newenvironment{skills}[0]{
  %%%%%%%%%%%%%%%%%%%%%%%%%%%%%%
  \resheading{\skillHeading}
  %%%%%%%%%%%%%%%%%%%%%%%%%%%%%%
  \begin{itemize}
}
{
  \end{itemize}
}
\newcommand{\skillItem}[2]{
  \item
    #1
    \begin{itemize}
      #2
    \end{itemize}
}

%Interests
\newenvironment{interests}[0]{
%%%%%%%%%%%%%%%%%%%%%%%%%%%%%%
\resheading{\interestHeading}
%%%%%%%%%%%%%%%%%%%%%%%%%%%%%%
\begin{itemize}
}
{
\end{itemize}
}
% Item
% 1: Area
% 2: Items
\newcommand{\interestItem}[2]{
  \item{
    {\bf #1:}#2
  }
}

%Referees
\newenvironment{referees}[0]{
%%%%%%%%%%%%%%%%%%%%%%%%%%%%%%
\resheading{\refereeHeading}
%%%%%%%%%%%%%%%%%%%%%%%%%%%%%%
\begin{itemize}
}
{
\end{itemize}
}
% Item
\newcommand{\refereeItem}[2]{
  \item{
    \href{#1}{#2}
  }
}
%-----------------------------------------------------------
