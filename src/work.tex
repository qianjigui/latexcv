% Work Experience
\begin{works}
\setLanguageValues[\nowLanguage]{

  %\workItem{XI'AN HENGSHENG CO.LTD}{Xian, China}{Systems architect}{Jul. 2009 - Oct. 2010(Part Time)}{
  %  \resitem{Designed the device which is the Machine Vision Based Intelligent Adapter for electrical and electronic circuit testing.}
  %  \resitem{Developed the protocol between each part.}
  %  \resitem{Developed the pattern recognition for pinholes in Linux with GCC/C \& Java.}
  %}

  \workItem{Cofly meeting support company}{Nanjing, China}{Software Engineer}{Mar. 2009 - Jun. 2010(Part Time)}
  {
    \resitem{Designed and developed a Web System(RoR) which is used for submitting papers, checking and announcing results.}
    \resitem{Designed and developed the automatic deployment system for each meeting.}
    \resitem{Administered the Running Server(Ubuntu 9.04) with HTTP, FTP, SMTP services.}
    \resitem{Managed and trained the technique team.}
  }

  %\workItem{International School of Software}{Wuhan University, China}{Software Engineer}{Mar. 2008 - Sep. 2008(Voluntary)}
  %{
  %  \resitem{Designed and developed a Web System(RoR) which is used for students' enrolling in ISS.}
  %}

  %\workItem{www.ziqiang.net}{Wuhan University, China}{Developer}{Jun. 2007 - Jun. 2008(Voluntary)}
  %{
  %  \resitem{Developed the Version 2 ziqiang.ent web(Jsp/Servlet) which is the students Web for supporting news, BBS, Videos, Download}
  %}
}{

  %\workItem{西安恒生自动化股份有限公司}{中国西安}{系统架构师}{2009.07 - 2010.10(兼职)}{
  %  \resitem{构架电缆智能测试应用软件}
  %  \resitem{开发了软件识别与硬件控制交互协议}
  %  \resitem{在Linux平台下开发完成了导线模式识别程序}
  %}

  \workItem{云OS系统集成开发}{云OS 系统集成}{资深开发工程师}{2013.07 - 2013.12}{
    \resitem{整合公司资源,打通BUG管理与公司联系人相关平台, 为产品集成与BUG分发跟踪与处理提供了一体化方案. 方便了集成工作与PM推进工作.}
    \resitem{开发了一套用于编译期条件编译支持的框架, 用于解决系统多版本支持的代码组织与管理问题.}
    \resitem{在系统FOTA升级方面, 构建了一套Patch处理框架, 实现了Patch程序的共享与独立性, 并方便开发与扩展. 解决了系统大版本升级的遗留问题. }
    \resitem{为系统对外开放, 构建了一套自动化的二进制集成管理与生成方案. 其中涉及模块动态组装与固化模块的版本管理机制. 简化了SCM的管理成本, 加快了系统移植速度}
  }

  \workItem{嵌入式虚拟机开发}{云OS 系统开发}{开发工程师}{2011.12 - 2013.06}
  {
    \resitem{重构了系统内存管理相关模块, 清晰了内存管理模块与其它系统的API关系, 制定了完整的虚拟机版本规范与自动发布上线流程和工具}
    \resitem{根据系统性能分析对内存管理模块进行了重新的功能造型与开发, 从移动内存管理变成MarkSweep模式}
    \resitem{基于MarkSweep模式进行了并行GC实现, 并根据系统特点与内存数据状况增加了相关内存优化机制. 开发中形成了系统内存分析的一套工具与方法}
    \resitem{负责系统兼容性与稳定性问题, 构建并制定了一套系统兼容性分析思路与方法. 解决了应用中心大量兼容性问题}
  }

  \workItem{Cofly学术会议服务公司}{中国南京}{软件开发}{2009.03 - 2010.06(学生兼职)}
  {
    \resitem{设计开发了公司第一代投稿审稿服务系统整体解决方案(基于RoR)}
    \resitem{管理公司所有服务器:Web,SMTP}
    \resitem{管理并培训公司技术团队}
  }

  %\workItem{国际软件学院}{武汉大学}{软件工程师}{2008.03 - 2008.05(义工)}
  %{
  %  \resitem{设计并开发了武汉大学南望山校区招生报名系统}
  %}

  %\workItem{www.ziqiang.net}{武汉大学}{开发人员}{2007.07 - 2008.06(义工)}
  %{
  %  \resitem{设计并开发了自强网络基础软件包与在线视频系统}
  %}
}

\end{works}
